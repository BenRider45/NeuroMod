\documentclass[11pt,letterpaper]{article}

% Packages
\usepackage[margin=1in]{geometry}
\usepackage{fancyhdr}
\usepackage{titlesec}
\usepackage{enumitem}
\usepackage{hyperref}
\usepackage{listings}
\usepackage{color}
\usepackage{graphicx}

\graphicspath{ {./images/} }

\lstset{frame=tb,
  language=Java,
  aboveskip=3mm,
  belowskip=3mm,
  showstringspaces=false,
  columns=flexible,
  basicstyle={\small\ttfamily},
  numbers=none,
  numberstyle=\tiny\color{gray},
  keywordstyle=\color{blue},
  commentstyle=\color{dkgreen},
  stringstyle=\color{mauve},
  breaklines=true,
  breakatwhitespace=true,
  tabsize=3
}




\NewDocumentCommand{\codeword}{v}{%
\texttt{\textcolor{black}{#1}}%
}
% Header setup
\pagestyle{fancy}
\fancyhf{}
\lhead{Ben Rider}
\rhead{\today}
\cfoot{\thepage}

% Title formatting
\titleformat{\section}{\large\bfseries}{\thesection}{1em}{}
\titleformat{\subsection}{\normalsize\bfseries}{\thesubsection}{1em}{}

\begin{document}

% Title
\begin{center}
    \LARGE\textbf{Progress Report} \\[0.5em]
    \large [Two-Component HVC(RA) Neuron Modeling System]
\end{center}

\vspace{1em}

\section*{Summary}
[I have designed and began implementation of a general c++ system strucutre in order to model systems of ODEs. This system also includes the functionality to generate octave scripts to export data for visualization. Furthermore, I have implemented classes in order to model the two-compartment HVC(RA) neuron model, as well as the Integrate-and-Fire model for testing purposes.
Unfortunately, I have run into some issues which verification of my implementation of the 4th-order Runge-Kutta method for solving the represented ODE systems.  ]

\section*{Completed Tasks}
\begin{itemize}[leftmargin=*]
  \item Implemented HVC(RA) neuron model
    \item Implemnnted Integrate-and-Fire neuron model
    \item Implemented the data export framework Exporto.

\end{itemize}

\section*{Current Work}
\begin{itemize}[leftmargin=*]
    \item Verification of RK4 method/neuron models.
    \item Further revisions to Exporto
\end{itemize}

\section*{Upcoming Tasks}
\begin{itemize}[leftmargin=*]
    \item Implement ability to construct neuron networks.
    \item Replicate HVC(RA) Figures presented in Jin 2009
\end{itemize}

\section*{Challenges and Issues}
The main challenge I have arrived at is in the verification of my implementation of the 4-th order Runge Kutta method.
For my implementation I adapt the code included in Ch.16 of A First Course in Nunerical Methods (Ascher, Uri M., Greif, Chen)
\begin{lstlisting}
% Integrate
for i=1:N
% Calculate the four stages
K1 = feval(f, t(i),y(:,i) );
K2 = feval(f, t(i)+.5*h, y(:,i)+.5*h*K1);
K3 = feval(f, t(i)+.5*h, y(:,i)+.5*h*K2);
K4 = feval(f, t(i)+h, y(:,i)+h*K3 );
% Evaluate approximate solution at next step
y(:,i+1) = y(:,i) + h/6 *(K1+2*K2+2*K3+K4);
end
\end{lstlisting}

I have used this section of their code to implement the step portion of the RK4 method. After not obtaining proper stabilization results from initial simultations of the HVC(RA) model, I began exploring verification of this method via use of the simpler Integrate-and-Fire method.


\section*{Current Results}

\includegraphics[scale = .25]{HVCRA_Results.jpg}\\
\includegraphics[scale = .25]{All_Gating_vars_Vs_Time.jpg}\\
\includegraphics[scale = .25]{Gating_Vars_H_C_Vs_Time.jpg}\\
\includegraphics[scale = .25]{Ca_Conc_Vs_Time.jpg}


\section*{Error Verification Via Integrate-and-Fire Model}

In order to attempt to diagnose the problems within this system. I first attempted to verify the error of my method via simulating the much simpler IF neuron model as described in the Training Module Assignment.
# Model
- This consists of one differential equation for membrane potential $\tau \frac{dV}{dt} = L - V + I_0(1 + \sin{t})$. $L= -70 mV$ being the resting membrane potential, $I_0 = 5mV$ being the injected current, $\tau = 20ms$ being our time constant. 
- We simulate this model for 40ms with the initial condition $V(0) = L$ for timesteps $.01, .02, .04, .1 ,.2, .001 ms$.
- We then plot the error of the final value of $V$ between the larger timesteps and $.001ms $ on a loglog graph and fit a linear function to the graph.


\includegraphics[scale =.25]{RK4 Implementation Error .jpg}

- We should expect our error to follow a linear pattern on a log scale, but that is not what we see in this case
\section*{Notes}
- Something about the gating variables (specifically $R,N$) having such large values compared to $H,C$ does not seem right.
\end{document}
